\documentclass{article}

\usepackage{amsmath, amsthm, amssymb, amsfonts}
\usepackage{thmtools}
\usepackage{graphicx}
\usepackage{setspace}
\usepackage{geometry}
\usepackage{float}
\usepackage{hyperref}
\usepackage[utf8]{inputenc}
\usepackage[english]{babel}
\usepackage{framed}
\usepackage[dvipsnames]{xcolor}
\usepackage{tcolorbox}

\colorlet{LightGray}{White!90!Periwinkle}
\colorlet{LightOrange}{Orange!15}
\colorlet{LightGreen}{Green!15}

\newcommand{\HRule}[1]{\rule{\linewidth}{#1}}

\declaretheoremstyle[name=Theorem,]{thmsty}
\declaretheorem[style=thmsty,numberwithin=section]{theorem}
\tcolorboxenvironment{theorem}{colback=LightGray}

\declaretheoremstyle[name=Proposition,]{prosty}
\declaretheorem[style=prosty,numberlike=theorem]{proposition}
\tcolorboxenvironment{proposition}{colback=LightOrange}

\declaretheoremstyle[name=Principle,]{prcpsty}
\declaretheorem[style=prcpsty,numberlike=theorem]{principle}
\tcolorboxenvironment{principle}{colback=LightGreen}

\setstretch{1.2}
\geometry{
    textheight=9in,
    textwidth=5.5in,
    top=1in,
    headheight=12pt,
    headsep=25pt,
    footskip=30pt
}

% ------------------------------------------------------------------------------

\begin{document}

% ------------------------------------------------------------------------------
% Cover Page and ToC
% ------------------------------------------------------------------------------

\title{ \normalsize \textsc{}
		\\ [2.0cm]
		\HRule{1.5pt} \\
		\LARGE \textbf{\uppercase{Relatório do trabalho\\ FARMÁCIA}
		\HRule{2.0pt} \\ [0.6cm] \LARGE{Ferramentas da Internet} \vspace*{8\baselineskip}}
		}
\date{}
\author{\textbf{Autores:} \\
		Isadora Aparecida Costa\\
		João Pedro Mendes Gerçóssimo\\
        Leonardo Pereira Farias\\
        Pedro Henrique Sampaio\\
        Pedro Paulo Carvalho Magalhães\\
        Ryan Henrique dos Reis Xavier\\
        10/04/2025}

\maketitle
\newpage

% ------------------------------------------------------------------------------

\section{Introdução}
Uma farmácia precisa de um banco de dados para gerenciar as informações de venda de produtos, o atendimento dos clientes, os agendamentos de serviços dentro da farmácia, como medir pressão e aplicar vacinas, a entrada de mercadorias por meio de compras e o estoque de produtos. A farmácia vende múltiplos produtos, como os remédios, os itens de saúde, os itens de higiene e cosméticos. Cada produto tem um código único que o identifica no sistema.  Além disso é atribuído o nome, a descrição, o fabricante, o preço, a quantidade em estoque e a data de validade.Todos esses atributos são simples, monovalorados e armazenados, menos a quantidade em estoque, que pode ser derivada com base nas compras e venda, e os fornecedores que são multivalorados, já que pode haver mais de um .Sua chave primária (CP) é o código do produto. Dentro da entidade Produto, existe uma especialização chamada medicamento, que apresenta os remédios controlados.Os medicamentos têm atributos específicos como a tarja (vermelha, preta, etc.) e se precisam de receita. Os clientes da farmácia podem ser pessoas físicas ou pessoas jurídicas, como clínicas e empresas. Essa é uma hierarquia de generalização. A entidade Cliente tem um identificador único chamado id cliente que é a chave primária, e possui os atributos telefone e e-mail (simples, monovalorados, armazenados). As pessoas físicas têm nome completo, CPF e data de nascimento; as pessoas jurídicas têm a razão social, CNPJ e nome do responsável. A farmácia registra as vendas e cada venda tem um código identificador (chave primária), uma data, o valor total (pode ser derivado da soma dos itens), e está ligada a um cliente e a um funcionário. Todos esses atributos são simples e armazenados. Uma venda pode conter vários produtos, e para isso existe a entidade intermediária chamada ItemVenda, que registra produto por produto dentro da venda, indicando a quantidade vendida e o preço unitário no momento. A chave primária composta dessa entidade é formada pelo código da venda e pelo código do produto. Todos os atributos são simples, monovalorados e armazenados. A farmácia também oferece serviços como por exemplo a aplicação de injeções. Os clientes podem agendar esses serviços por meio da entidade Agendamento. Cada agendamento possui um código, a chave primária, data, horário, tipo de serviço, o funcionário que fará o atendimento, e o cliente que solicitou. Todos os atributos são simples, monovalorados e armazenados. A farmácia realiza compras para repor seu estoque. Cada compra possui um código (chave primária), data, valor total (derivado) e é feita com um fornecedor específico. A entidade ItemCompra registra os produtos comprados em cada compra, incluindo quantidade adquirida e valor pago pela unidade. Sua chave primária é composta (código da compra e código do produto), e os atributos são simples, monovalorados e armazenados. Os fornecedores são as empresas que vendem produtos para a farmácia. Cada fornecedor tem um CNPJ (chave primária), nome e telefone (simples, monovalorados, armazenados). Um fornecedor pode estar associado a várias compras. Por fim, os funcionários da farmácia são registrados com um código (chave primária), nome, CPF, telefone, cargo e salário. Todos os atributos são simples, monovalorados e armazenados. Um funcionário pode realizar várias vendas ou atender vários agendamentos. 

\newpage


\end{document}