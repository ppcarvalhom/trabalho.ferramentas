\documentclass{article}

\usepackage{amsmath, amsthm, amssymb, amsfonts}
\usepackage{thmtools}
\usepackage{graphicx}
\usepackage{setspace}
\usepackage{geometry}
\usepackage{float}
\usepackage{hyperref}
\usepackage[utf8]{inputenc}
\usepackage[english]{babel}
\usepackage{framed}
\usepackage[dvipsnames]{xcolor}
\usepackage{tcolorbox}

\colorlet{LightGray}{White!90!Periwinkle}
\colorlet{LightOrange}{Orange!15}
\colorlet{LightGreen}{Green!15}

\newcommand{\HRule}[1]{\rule{\linewidth}{#1}}

\declaretheoremstyle[name=Theorem,]{thmsty}
\declaretheorem[style=thmsty,numberwithin=section]{theorem}
\tcolorboxenvironment{theorem}{colback=LightGray}

\declaretheoremstyle[name=Proposition,]{prosty}
\declaretheorem[style=prosty,numberlike=theorem]{proposition}
\tcolorboxenvironment{proposition}{colback=LightOrange}

\declaretheoremstyle[name=Principle,]{prcpsty}
\declaretheorem[style=prcpsty,numberlike=theorem]{principle}
\tcolorboxenvironment{principle}{colback=LightGreen}

\setstretch{1.2}
\geometry{
    textheight=9in,
    textwidth=5.5in,
    top=1in,
    headheight=12pt,
    headsep=25pt,
    footskip=30pt
}

% ------------------------------------------------------------------------------

\begin{document}

% ------------------------------------------------------------------------------
% Cover Page and ToC
% ------------------------------------------------------------------------------

\title{ \normalsize \textsc{}
		\\ [2.0cm]
		\HRule{1.5pt} \\
		\LARGE \textbf{\uppercase{Relatório do trabalho\\ CONCESSIONÁRIA}
		\HRule{2.0pt} \\ [0.6cm] \LARGE{Ferramentas da Internet} \vspace*{9.5\baselineskip}}
		}
\date{}
\author{\textbf{Autores:} \\
		Isadora Aparecida Costa\\
		João Pedro Mendes Gerçóssimo\\
        Leonardo Pereira Farias\\
        Pedro Henrique Sampaio\\
        Pedro Paulo Carvalho Magalhães\\
        Ryan\\
        10/04/2025}

\maketitle

% ------------------------------------------------------------------------------

\section{Introdução}
Uma concessionária precisou de um banco de dados para gerenciar as suas informações. A mesma trabalha com vendas de veículos, serviços de manutenção, atendimento ao cliente, e também possui o controle das peças e dos fornecedores. O sistema vai resolver e ajudar tudo isso.  
Os veículos que foram vendidos pela loja possuem um código único, e também informações de marca, ano, modelo, cor, tipo do combustível, quilometragem e o valor. Cada veículo pode estar com a seguinte situação/status: disponível, reservado e vendido.
Existem dois tipos de clientes: pessoa física ou pessoa jurídica. Pessoas físicas são os clientes comuns, com nome, CPF, telefone e e-mail. E as pessoas jurídicas são empresas, que têm CNPJ, nome do responsável e telefone. Mesmo sendo tipos diferentes, os dois são tratados como clientes no sistema. Um cliente pode comprar mais de um veículo. Toda venda tem um código, a data da venda, o valor final, o veículo vendido e o funcionário que fez a venda. Esse funcionário também está no sistema com informações como nome, CPF, telefone, cargo e salário. Os funcionários podem atuar nas vendas ou nos serviços.
Além de vender carros, a concessionária também faz serviços de manutenção, como troca de óleo, revisão, conserto de peças, entre outros. Para organizar esses atendimentos, o cliente precisa marcar um horário, e esse agendamento é registrado no sistema.
Quando ele faz o agendamento, o sistema guarda informações como data e o horário em que o serviço vai ser feito, o tipo de serviço que o cliente solicitou (revisão ou troca de pneus), o veículo que será atendido, o funcionário que ficará responsável por fazer o serviço e o cliente que solicitou o atendimento.
Durante o serviço, pode ser que use as peças que estão no estoque da concessionária, como filtros, óleo, pastilhas de freio, entre outras. E essas peças também são cadastradas no sistema com algumas informações importantes, cada peça tem um código de identificação,nome, a quantidade disponível no estoque, o preço e para quais veículos ela serve.
Peças usadas nos serviços ou mantidas no estoque da concessionária são compradas de fornecedores, que são empresas cadastradas no sistema com informações como CNPJ, nome e telefone. Quando a concessionária realiza a compra de peças, essa compra é registrada no sistema com um código, a data em que foi feita, o valor total e os detalhes das peças compradas.
Como uma mesma compra pode incluir várias peças diferentes, é necessário um controle mais especificado para registrar cada item comprado separadamente. Para isso, existe uma parte do sistema chamada "Lista de itens comprados". Ela serve para mostrar dentro de cada compra quais peças foram adquiridas, quantas unidades de cada peça foram compradas e qual era o preço de cada uma no momento da compra. Assim a Lista de itens comprados funciona como uma tabela intermediária, que liga a compra com as peças compradas informando item por item.


\newpage


\end{document}